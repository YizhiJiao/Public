%This template is based on one provided by the American Physical Society for submission to its journals.

\documentclass[aps,twocolumn,showpacs,preprintnumbers]{revtex4}

%The following packages add LaTeX commands that make formatting and writing math easier

\usepackage{graphicx}  % Include figure files
\usepackage{subfigure}

\linespread{1.1}
\usepackage{fancyhdr}
\usepackage{parskip}
\usepackage[T1]{fontenc}
\usepackage{dcolumn}   % Align table columns on decimal point

\usepackage{bm}        % bold math
\usepackage{amsfonts}  % Common math fonts
\usepackage{amsmath}   % Common math functions
\usepackage{amssymb}   % Common math symbols

%The following custom commands simplify commonly used LaTeX commands

\newcommand{\pic}[2]{\begin{center} \includegraphics[scale=#1]{#2}\end{center}}
\newcommand{\re}[1]{\mathrm{Re}\left(#1\right)}
\newcommand{\im}[1]{\mathrm{Im}\left(#1\right)}
\newcommand{\bdot}[1]{\dot{ \bb {#1}}}
\newcommand{\bddot}[1]{\ddot{ \bb {#1}}}
\newcommand{\bidot}[1]{\dot{ \bi{ #1}}}
\newcommand{\biddot}[1]{\ddot{ \bi {#1}}}
\newcommand{\ep}{\varepsilon}
\newcommand{\for}{\quad \quad \mathrm{for} \quad\quad}
\newcommand{\then}{\quad \quad \implies \quad\quad}
\newcommand{\an}{\quad \quad \mathrm{and} \quad\quad}
\newcommand{\ifff}{\quad \quad \mathrm{if} \quad\quad}
\newcommand{\where}{\quad \quad \mathrm{where} \quad\quad}
\newcommand{\dg}{^\dagger}
\newcommand{\semi}{\quad \quad \mathrm{;} \quad\quad}
\newcommand{\paren}[1]{\left( #1 \right)}
\newcommand{\brac}[1]{\left[ #1 \right]}
\newcommand{\bra}[1]{\left\langle #1 \right|}
\newcommand{\exv}[1]{\left\langle #1 \right\rangle}
\newcommand{\pwisein}{\left\{ \begin{array}{ll}}
\newcommand{\pwiseout}{\end{array}\right.}
\newcommand{\ket}[1]{\left| #1 \right\rangle}
\newcommand{\bracket}[2]{\left\langle #1 | #2 \right\rangle}
\newcommand{\trace}[1]{\mathrm{Tr} \left( #1 \right)}
\renewcommand{\det}[1]{\mathrm{det}\left( #1 \right)}
\newcommand{\del}[1]{\frac{\partial}{\partial #1}}
\newcommand{\fulld}[1]{\frac{d}{d #1}}
\newcommand{\fulldd}[2]{\frac{d #1}{d #2}}
\newcommand{\dell}[2]{\frac{\partial #1}{\partial #2}}
\newcommand{\delltwo}[2]{\frac{\partial^2 #1}{\partial #2 ^2}}
\newcommand{\bb}{\mathbf}
\newcommand{\bi}{\boldsymbol}
\newcommand{\eq}[1]{\begin{equation} #1 \end{equation}}
\newcommand{\radhalf}{ \frac{ \sqrt{2}}{2}}
\newcommand{\sigx}{\left( \begin{array}{cc} 0 & 1\\ 1 & 0 \end{array}\right)}
\newcommand{\sigy}{\left( \begin{array}{cc} 0 & -i\\ i & 0 \end{array}\right)}
\newcommand{\sigz}{\left( \begin{array}{cc} 1 & 0\\ 0 & -1 \end{array}\right)}
\renewcommand{\matrix}[1]{\left( \begin{array} #1 \end{array}\right)}
\newcommand{\thermo}[3]{\left( \frac{\partial #1}{\partial #2} \right)_{#3}}
\newcommand{\coolfrac}[2]{\left( \frac{ #1}{ #2} \right)}

%\setlength{\parskip}{\baselineskip}
\setlength{\parindent}{10pt}

\begin{document}

\title{Introduction - First draft}

\author{Yizhi Jiao$^\ast$}

\affiliation {\it Physics Department, University of California, Santa Barbara, CA 93106}

\date{submitted \today}


\maketitle %This command sets formatting, including a line for PACS numbers.  You don't need to know what PACS numbers are for now, but if you're curious, you can read about them here:  https://journals.aps.org/PACS

%P1: Big Picture
%% This paragraph should convey why what you are writing about is interesting. 
%% Refer to the example text below for a sense of scope and tone.  Then replace the example text with your own.
%% Note that it is helpful to start each sentence on a new line.  LaTeX will only generate a paragraph break after two consecutive line-breaks.

Investigating nuclear transitions, specifically the emission of highly energetic gamma rays, provides valuable insights into the composition and behavior of radioactive elements. 
The discovery of radioactivity by Becquerel in the late 19th century marked a significant milestone in the understanding of atomic and nuclear phenomena\cite{Becquerel}.
The unique energy spectra of gamma rays emitted during nuclear transitions are fascinating and have critical implications for various technological applications, such as radiation therapy in cancer treatment and nuclear energy production. 



%P2: Current Context: 
%% This paragraph should covey background necessary to understand what you are writing about.
%% It should introduce (and cite) specific work that lead to the present work.
Radioactive elements, such as Cobalt-60, undergo a series of decay processes, including beta decay and subsequent gamma emissions. 
Cobalt-60 decays into Nickel-60 and emits gamma rays with specific energies.
Thanks to the invention of Geiger counter\cite{Geiger}, the emitted energies can be measured by counting. 
This study will primarily concentrate on the relationship between the counts of the Geiger counter and the density thickness of the absorbers. 
%P3: Present Work
%% This paragraph should covey (i) what you did, (ii) how you did it, AND (iii) what you found.
%% There is no place of suspense in scientific writing.  Readers' interest comes from a desire to know your approach, see your data and understand your logic.

Here I report an improved approach of known method \cite{Manual} to measuring the wavelength of gamma rays emitted by Cobalt-60 using a Geiger counter with absorbers.
This approach yielded a value of $(blank)$ m.  
The precision of this measurement was limited by the uncertainty of the counts counted by the Geiger Counter.


\begin{thebibliography}{10}

% \bibitem{EinsteinPHYSREV1905/7} A. Einstein, (From ``The Collected Papers, Vol 2, The Swiss Years: Writings, 1900–1909'', English Translation) ``On the Electrodynamics of Moving Bodies'', {\it Annalen Der Physik} {\bf 17}, 891–921 (1905); and ``On the Inertia of Energy Required by the Relativity Principle'', {\it Annalen Der Physik} {\bf 23}, 371–384 (1907).

% \bibitem{TaylorWheeler92} E.F. Taylor \& J.A. Wheeler, {\it Spacetime Physics}, (WH Freeman, 1992).

% \bibitem{severalApplicationRefs} I.~M. Author1 \&  I.~M. Author2, Article One's Title, {\it Abbrev. J. Title} {\bf vol}, pagei-pagef (year); A.~O. Scientist, Article Two's Title, {\it Abbrev. J. Title} {\bf vol}, pagei-pagef (year).

% \bibitem{BobisJAHH} L. Bobis \& J. Lequeux,  ``Cassini, R\o mer and the velocity of light'', {\it J. Astro. Hist. and Herit.}, {\bf 11}, 97-105 (2008).

% \bibitem{EvensonPRL1972} K.~M. Evenson {\it et al.},  ``Speed of Light from Direct Frequency and Wavelength Measurements of the Methane-Stabilized Laser'', {\it Phys. Rev. Lett.} {\bf 29}, 1346-1349 (1972).

% \bibitem{refExample1} O. Author, ``Article Title'', {\it Abbrev. J. Title} {\bf vol}, pagei-pagef (year).

% \bibitem{refExample2} I.~M. Author1 \&  I.~M. Author2,  ``Article Title'', {\it Abbrev. J. Title} {\bf vol}, pagei-pagef (year).

% \bibitem{refExampleMANY} T.~F. Author {\it et al.},  ``Article Title'', {\it Abbrev. J. Title} {\bf vol}, pagei-pagef (year).

% \bibitem{comment} It is ok to include footnotes among the references, like this.

\bibitem{Becquerel} H. Becquerel, ``Sur les radiations {\'e}mises par phosphorescence'', {\it C. r. hebd. s{\'e}ances Acad. sci.} {\bf 122}, 420-421 (1896).
\bibitem{Geiger}H. Geiger \& W. M{\"u}ller, ``Elektronenz{\"a}hlrohr zur Messung schw{\"a}chster Aktivit{\"a}ten'', {\it Sci. Nat.} {\bf 16}, 617 (1928). 
\bibitem{Manual}D. Fygenson, ``Gamma Ray Absorption'' {\it Physics 25 Lab Manual},15-25(2021). 

\end{thebibliography}
\end{document}